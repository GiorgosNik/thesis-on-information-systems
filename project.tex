\documentclass{article}

\usepackage[greek,english]{babel}
\usepackage[utf8x]{inputenc}
\usepackage{natbib}
\usepackage{graphicx}
\usepackage{subfigure}
\graphicspath{ {.//} }
\usepackage{xcolor}
\graphicspath{ {./images/} }
\usepackage{titling}

\setlength{\droptitle}{10em}

\title{%
  \selectlanguage{greek}\Huge
 Μοντέλα υπολογιστικού νέφους \\}
  


\begin{document}
\selectlanguage{greek}

\author{\selectlanguage{greek}\LARGE
  Γεώργιος Νικολάου\\
   \texttt{\selectlanguage{english}\large sdi1800134}
  \and
  \selectlanguage{greek}\LARGE
  Νεφέλη Ταβουλάρη\\
   \texttt{\selectlanguage{english}\large sdi1800190}
}

\maketitle


\begin{figure}
\centering
\begin{subfigure}
  \centering
  \includegraphics[width=40mm]{dit_logo}
  \label{fig:sub1}
\end{subfigure}%
\begin{subfigure}
  \centering
  \includegraphics[width=50mm]{NKUA_logo}
  \label{fig:sub2}
\end{subfigure}
\label{fig:test}
\end{figure}




\newpage
\tableofcontents
\newpage
\section{Εισαγωγή}
\section{Σύγχρονα Πληροφοριακά Συστήματα}
\subsection{Ανάπτυξη}
\subsection{Νέες Ανάγκες}
\section{Υπολογιστικό Νέφος}
\subsection{Οφέλη}
\subsection{Προκλήσεις}
\section{Μορφές υπολογιστικού νέφους}
\newpage
\subsection{Λογισμικό ως Υπηρεσία}
Αναμφισβήτητα,η αγορά λογισμικού για τις επιχειρηματικές ανάγκες μιας εταιρείας αποτελεί μια κρίσιμη και περίπλοκη διαδικασία. Στόχος είναι πάντα η ελαχιστοποίηση του κόστους , αλλά και η αύξηση της παραγωγικότητας, επενδύοντας σε όσο μεγαλύτερους αυτοματισμούς. Συνηθισμένα παραδείγματα λογισμικού που χρησιμοποιούνται από τις επιχειρήσεις είναι τα εξής:
\begin{itemize}
  \item Λογισμικό εσωτερικής επικοινωνίας, πχ \selectlanguage{english} Microsoft Teams, Zoom, Webex, Outlook, Jira, Confluence, Slack\selectlanguage{greek}
  \item Λογισμικό για διαχείριση λογιστικών, φορολογικών διαδικασιών
  \item \selectlanguage{english} Enterprise Resource Planning programs (ERP)
  \item Office 365, \selectlanguage{greek}πχ \selectlanguage{english}Word, Excel, Access, Google Docs
  \item Portal \selectlanguage{greek} εταιρείας, διαχείριση αδειών, διαχείριση tickets στο \selectlanguage{english}ΙΤ \selectlanguage{greek}τμήμα \selectlanguage{english}
  \item Version Control System,\selectlanguage{greek} πχ \selectlanguage{english}Github, Gitlab, Bitbucket \selectlanguage{greek}
  \item Λογισμικό διαχείρισης Ανθρωπίνου Δυναμικού \selectlanguage{english}
  \item Customer Relationship Management (CRP), Marketing programs
  \item CAD \selectlanguage{greek}Λογισμικό\selectlanguage{english}
  \item Code Editor,\selectlanguage{greek} πχ\selectlanguage{english} VSCode, IntelliJ, PyCharm\selectlanguage{greek}

\end{itemize}
Παραδοσιακά, αυτή η ανάγκη των εταιρειών ικανοποιούνταν με το λεγόμενο, Λογισμικό ως Προϊόν \selectlanguage{english}(Software as a Product, SaaP), \selectlanguage{greek}βάσει του οποίου, η εταιρεία αγοράζει άδεια για να χρησιμοποιεί τη συγκεκριμένη υπηρεσία, την οποία φιλοξενεί ύστερα, στις εγκαταστάσεις της \selectlanguage{english}(on premises).\selectlanguage{greek} Επίσης, αναλαμβάνει τα έξοδα συντήρησης και ανανέωσης του λογισμικού. \\ \\
Ωστόσο, με την ανάπτυξη του Υπολογιστικού Νέφους, το Λογισμικό ως Υπηρεσία γίνεται όλο και πιο διαδεδομένο.Το Λογισμικό ως Υπηρεσία, γνωστό και ως \selectlanguage{english}Software as a Service (SaaS) \selectlanguage{greek}αναφέρεται στην παροχή ενός λογισμικού / πληροφοριακού συστήματος, το οποίο φιλοξενείται κεντρικά, στο Νέφος. Η φιλοσοφία πίσω από αυτή την διαδεδομένη μέθοδο, είναι πως ο χρήστης μπορεί να απολαμβάνει μια ηλεκτρονική υπηρεσία, χωρίς να χρειάζεται να διαχειρίζεται τις υποδομές, το δίκτυο, τις βάσεις δεδομένων, το ενδιάμεσο λογισμικό, τους διακομιστές / σέρβερς, τα δεδομένα, το λειτουργικό σύστημα, αλλά ούτε και τις εφαρμογές. Εν ολίγοις, ο χρήστης χρησιμοποιεί το Πληροφοριακό Σύστημα , ως συνδρομητής, για να ικανοποιήσει τις ανάγκες του και ο Πάροχος της υπηρεσίας αναλαμβάνει να διαχειρίζεται όλους τους απαιτούμενους πόρους.  \\ 
\newpageΤα κυριότερα πλεονεκτήματα ενός τέτοιου μοντέλου είναι τα εξής :
\begin{itemize}
    \itemΧαμηλότερο κόστος για τις επιχειρήσεις και τους ιδιώτες, οι οποίοι δε χρειάζεται να αγοράσουν το λογισμικό και τις απαραίτητες άδειες. Ανάπτυξη οικονομίας κλίμακας με μείωση κόστους συντήρησης και παροχής του λογισμικού. Επίσης, το κόστος είναι συγκεκριμένο \selectlanguage{english}(pay-as-you-go) \selectlanguage{greek}και πολλές φορές δεν υπάρχει από την αρχή της χρήσης (πχ \selectlanguage{english}Netflix).\selectlanguage{greek} 
    \itemΔεν υπάρχει ανάγκη γνώσης χρήσης των διαφόρων λογισμικών, ούτε επομένως και εξειδικευμένου προσωπικού σε μια επιχείρηση, με αποτέλεσμα ξανά τη μείωση κόστους. Ευχρηστία, σύνδεση στην εφαρμογή μέσω κάποιου \selectlanguage{english}Dashboard\selectlanguage{greek} ή \selectlanguage{english}API.\selectlanguage{greek}
    \itemΠρόσβαση από παντού, μέσω Διαδικτύου και οποιασδήποτε συσκευής, χωρίς να χρειάζεται να εγκατασταθεί πρόσθετο λογισμικό (πχ \selectlanguage{english}dependencies,\selectlanguage{greek} εργαλεία, packages, λειτουργικό σύστημα), ή σύνδεση με \selectlanguage{english}VPN\selectlanguage{greek}. Επιτυγχάνεται, λοιπόν, μεγαλύτερη ευελιξία και αύξηση της παραγωγικότητας 
    \itemΌλοι οι χρήστες έχουν την ίδια έκδοση, η οποία αναβαθμίζεται αυτόματα
    \itemΠρόκειται για ολοκληρωμένα και εύχρηστα Πληροφοριακά Συστήματα, ανεξάρτητα από το Λειτουργικό Σύστημα ή το Υλικό του καθενός, ακολουθούν επομένως τις αρχές της εικονικοποίησης (\selectlanguage{english}virtualization)\selectlanguage{greek} και της αφαιρετικότητας (\selectlanguage{english}abstraction)\selectlanguage{greek}
    \itemΑποτελεί μια πρακτική που μπορεί να αποβεί ωφέλιμη και για το περιβάλλον, καθώς χρησιμοποιούνται όσο το δυνατόν λιγότεροι πόροι 
    \itemΚαλύτερη συνεργασία μέσα σε μια εταιρεία, λόγω του \selectlanguage{english}multitenancy,\selectlanguage{greek} δηλαδή της παράλληλης χρήσης της ίδιας εφαρμογής από πολλούς χρήστες. Χαρακτηριστικό του \selectlanguage{english}Multitenancy\selectlanguage{greek} είναι πως οι χρήστες νιώθουν σα να έχουν αποκλειστική χρήση του Λογισμικού
    \itemΑσφάλεια από κακόβουλους εισβολείς, προστασία, απόδοση, προσβασιμότητα, υψηλή διαθεσιμότητα, κλιμάκωση, αξιοπιστία και γενικότερα, όλα τα γνωστά πλεονεκτήματα του\selectlanguage{english} Cloud Computing\selectlanguage{greek}. Καθώς οι διάφορες εφαρμογές φιλοξενούνται σε κορυφαία \selectlanguage{english}Data Centers,\selectlanguage{greek} ειδικά σχεδιασμένα με αυτό το σκοπό, σε υψηλού επιπέδου υποδομές, με \selectlanguage{english}firewalls,\selectlanguage{greek} κρυπτογράφηση και ελεγχόμενη πρόσβαση
    \itemΧρησιμοποιώντας \selectlanguage{english}SaaS, \selectlanguage{greek}μια επιχείρηση έχει την ευκαιρία να αφοσιωθεί σε αυτό που την ενδιαφέρει πιο πολύ, το προϊόν της. Το ίδιο βέβαια ισχύει και για τους ιδιώτες. Αύξηση, λοιπόν, της καινοτομίας και της ευκινησίας \selectlanguage{english}(agility)
\selectlanguage{greek}
\end{itemize}
\newpage
Τα κυριότερα μειονεκτήματα :
\begin{itemize}
 \itemΑναγκαία η χρήση Διαδικτύου, διαφορετικά δεν μπορεί να υπάρξει πρόσβαση στην εφαρμογή
\itemΧαμηλή ασφάλεια ευαίσθητων δεδομένων, αν δεν είναι έμπιστος ο πάροχος
\itemΈλλειψη ελέγχου
\itemΤαχύτητα μετάδοσης
\end{itemize}
Συμπέρασμα : \\ \\
Με τη σωστή επιλογή ενός γνωστού και αποδεκτού παρόχου υπηρεσιών λογισμικού, την υπογραφή Συμφωνητικού για το Επίπεδο Λειτουργίας Υπηρεσίας,\selectlanguage{english} SLA (service level agreement),\selectlanguage{greek} όπου καταγράφονται όλες οι νομικές υποχρεώσεις του παρόχου, καθώς και με τη χρήση\selectlanguage{english}  5G,\selectlanguage{greek} το Λογισμικό ως Υπηρεσία είναι μια φτηνή και σίγουρη επιλογή για μια επιχείρηση, ειδικά αν δεν αναζητά κάποια εξειδικευμένη υπηρεσία, όπου έχει ανάγκη\selectlanguage{english}  customization.\selectlanguage{greek}
\subsection{Υποδομή ως Υπηρεσία}

\newpage
\subsection{Πλατφόρμα ως Υπηρεσία}

\section{Από το \selectlanguage{english}IaaS \selectlanguage{greek}στο \selectlanguage{english}SaaS}

\subsection{Σύγκριση}
\section{Συμπεράσματα - Επισημάνσεις}
\end{document}
